\documentclass[a4paper,11pt]{article}
\usepackage{CJKutf8} %支持中文
\usepackage[T1]{fontenc}
\usepackage[utf8]{inputenc}
\usepackage{lmodern}
\usepackage{listings}
\usepackage{amsmath} %提供\begin{equation}
\usepackage{amssymb} %提供\substack
\usepackage{latexsym}
\usepackage{cite}
\usepackage{lipsum} 
\usepackage{enumitem}
\usepackage{tikz} %
\usepackage{bm} %在需要加粗的数学公式前使用\bm命令
%\usepackage{ctex}

%\usepackage{xeCJK}

\usepackage{polynom} %提供\polylongdiv
\input{longdiv.tex} %竖式除法longdiv
%\input{split.tex}
      
\title{title 中文标题显示??? }
\author{pchaos中文字}

%begin 选择题选项排版
\usepackage{graphicx}%以下代码不依赖宏包,加它纯粹为演示图形选项
%预备
\newlength\lxxmax
\newlength\lhalf
\newlength\lhalfhalf
\newsavebox\xxboxa
\newsavebox\xxboxb
\newsavebox\xxboxc
\newsavebox\xxboxd
\newcommand\getboxsandmax[4]{%存放内容并获取最大宽
\sbox\xxboxa{#1}%
\sbox\xxboxb{#2}%
\sbox\xxboxc{#3}%
\sbox\xxboxd{#4}%
\ifdim\wd\xxboxa>\wd\xxboxb
\lxxmax\wd\xxboxa\relax
\else
\lxxmax\wd\xxboxb\relax
\fi
\ifdim\lxxmax<\wd\xxboxc
\lxxmax\wd\xxboxc\relax
\fi
\ifdim\lxxmax<\wd\xxboxd
\lxxmax\wd\xxboxd\relax
\fi
}

%四选项自动排版,用法 \xx{选项}{选项}{选项}{选项}
\newcommand\xx[4]{\par
\getboxsandmax{A. #1}{B. #2}{C. #3}{D. #4}%
\addtolength\lxxmax{1em}%
\setlength\lhalf{\dimexpr(\linewidth-\parindent)/2\relax}%
\setlength\lhalfhalf{.5\lhalf}%
\ifdim\lxxmax>\lhalf
    A. #1 \par
    B. #2 \par
    C. #3 \par
    D. #4
\else
    \ifdim\lxxmax>\lhalfhalf
        \begin{minipage}{\lhalf}
        \usebox\xxboxa
        \end{minipage}%
        \begin{minipage}{\lhalf}
        \usebox\xxboxb
        \end{minipage}\par
        \begin{minipage}{\lhalf}
        \usebox\xxboxc
        \end{minipage}%
        \begin{minipage}{\lhalf}
        \usebox\xxboxd
        \end{minipage}%
    \else
        \begin{minipage}{\lhalfhalf}
        \usebox\xxboxa
        \end{minipage}%
        \begin{minipage}{\lhalfhalf}
        \usebox\xxboxb
        \end{minipage}%
        \begin{minipage}{\lhalfhalf}
        \usebox\xxboxc
        \end{minipage}%
        \begin{minipage}{\lhalfhalf}
        \usebox\xxboxd
        \end{minipage}%
    \fi
\fi
}

%四图片选项专用(每图不得太宽)
\newcommand\fourtuxx[4]{\par\vspace{2ex}%
\getboxsandmax{#1}{#2}{#3}{#4}%
\addtolength\lxxmax{1em}%
\setlength\lhalf{.5\linewidth}%
\setlength\lhalfhalf{.5\lhalf}%
\ifdim\lxxmax>\lhalf
images too big! please change width to less than 0.5linewidth-1em.
\else
    \ifdim\lxxmax>\lhalfhalf
        \noindent
        \begin{minipage}[b]{\lhalf}
        \centering
        \usebox\xxboxa\par A
        \end{minipage}%
        \begin{minipage}[b]{\lhalf}
        \centering
        \usebox\xxboxb\par B
        \end{minipage}\par\vspace{2ex}%
        \noindent
        \begin{minipage}[b]{\lhalf}
        \centering
        \usebox\xxboxc\par C
        \end{minipage}%
        \begin{minipage}[b]{\lhalf}
        \centering
        \usebox\xxboxd\par D
        \end{minipage}%
    \else
        \noindent
        \begin{minipage}[b]{\lhalfhalf}
        \centering
        \usebox\xxboxa\par A
        \end{minipage}%
        \begin{minipage}[b]{\lhalfhalf}
        \centering
        \usebox\xxboxb\par B
        \end{minipage}%
        \begin{minipage}[b]{\lhalfhalf}
        \centering
        \usebox\xxboxc\par C
        \end{minipage}%
        \begin{minipage}[b]{\lhalfhalf}
        \centering
        \usebox\xxboxd\par D
        \end{minipage}%
    \fi
\fi
}
%end 选择题选项排版

% begin tikz画带圈数字
\newlength\szg
\newcommand\quan[1]{%
\settoheight\szg{#1}%
\tikz[baseline]{\pgfmathparse{
ifthenelse(#1 < 10, 1, ifthenelse(#1 < 100, 0.75, 0.5))
}
\let\hfs\pgfmathresult
\node at (0,\szg/2) {\makebox[0em][c]{\scalebox{\hfs}[1]{#1}}};
\draw (0,\szg/2) circle (\szg/2+0.35ex);
}}
% end tikz画带圈数字

\begin{document}
\begin{CJK}{UTF8}{gkai}
\maketitle
\tableofcontents

\newpage
\begin{abstract}
摘要
\end{abstract}

\newpage
\section{Introduction}
引言\cite{sample1}


\section{Methodologies}
\begin{equation}
	f(x) = \sum_{i=1}^{n} {x_i}
\end{equation}

\begin{equation}
	f(x) = x_1 + x_2 + x_3 + ...... + x_n
\end{equation}

\begin{equation}
	f(x) = \int_{i=1}^{n} {x_i}
\end{equation}

\begin{equation}
	X=
	\begin{cases}
	5, \text{if X is divisible by 5}
	\\
	10, \text{if X is divisible by 10}
	\\
	-1, \text{otherwise}
	\end{cases}
\end{equation}

\begin{equation}
	X = 
	\frac{\substack{\sum{X_i}}}
	{\substack{\sum{X_j}}}
\end{equation}

\begin{equation}
	X \leqslant y %x<=y
\end{equation}

\begin{equation}
	X \geqslant y %x>=y
\end{equation}

\[ E=mc^2 \]
$8\div4=2$

$8\div4=2$

  $ 8\div 4\equiv 2 $
\begin{math}
 7.8\div1.6 \alpha( < > = ) 78 \div16 \end
 {math} 
 
分数
$ \frac{1}2{} \dfrac{1}{3}$

代数
$ x = a_0 + \frac{1}{a_1 + \frac{1}{a_2 + \frac{1}{a_3 + a_4}}} $

微积分 $\frac{d}{dx}\ln(x)=\frac{1}{x}$

几何
$ \widehat{AB} $

三角
$\left(\frac{\pi}{2}-\theta \right )$

矩阵
$\begin{pmatrix}
 a_{11} & a_{12} & a_{13}\\ 
 a_{21} & a_{22} & a_{23}\\ 
 a_{31} & a_{32} & a_{33}
 \end{pmatrix}$
 
 多行  \begin{gather}
 5^2+12^2=13^2\\
 a^2 + b^2 = c^2
 \end{gather}
 
 无编号 \begin{gather}
 5^2+12^2=13^2 \notag\\
 a^2 + b^2 = c^2\notag
\end{gather}
 
表格
\begin{tabular}{l c c  c r}
姓名 & 语文 & 数学 & 外语 & 备注 \\
张三 & 87 & 100 & 93 & 优秀  \\
李四 & 75 & 64 & 52 & 补考另行通知  \\
张三 & 80 & 82 & 78 &   \\
\end{tabular}

\begin{tabular}{|l| c| c|  c| r|}
\hline
姓名 & 语文 & 数学 & 外语 & 备注 \\
\hline
张三 & 87 & 100 & 93 & 优秀  \\
\hline
李四 & 75 & 64 & 52 & 补考另行通知  \\
\hline
张三 & 80 & 82 & 78 &   \\
\hline
\end{tabular}

竖式加减法
$\begin{array}{ccc} 
&A&B\\ 
+&C&D\\ 
\hline 
=&I&I 
\end{array}$

竖式乘法
$\begin{array}{ccc} 
&AB\\ 
\times &CD\\ 
\hline 
=I&II 
\end{array}$

自定义列表序号
\makeatletter 
\def\zdyxh#1{\expandafter\@zdyxh\csname c@#1\endcsname} 
\def\@zdyxh#1{% 
\ifcase#1\or 自\or 定\or 义\or 序\or 号\else 超了\fi} 
\makeatother 
\AddEnumerateCounter*{\zdyxh}{\@zdyxh}{1} 
% 没搞懂下面这句是干啥用的
\lipsum[101] 
\begin{enumerate}[label=\zdyxh*.] 
\item 内容 
\item 内容 
\item 内容 
\item 内容 
\item 内容 
\item 内容 
\item 内容 

%\lipsum[101] 
\end{enumerate}


选择题选项排版
\newcommand\test{test test test test test test test test test test test test test test test test test test test test test test}

\rule{0.15\linewidth}{1ex} \test
\xx{\rule{0.15\linewidth}{1ex}}234

\rule{0.3\linewidth}{1ex} \test
\xx{\rule{0.3\linewidth}{1ex}}234

\rule{0.5\linewidth}{1ex} \test
\xx{\rule{0.5\linewidth}{1ex}}234

\begin{itemize}
\item\rule{0.15\linewidth}{1ex} \test
\xx{\rule{0.15\linewidth}{1ex}}234

\item\rule{0.3\linewidth}{1ex} \test
\xx{\rule{0.3\linewidth}{1ex}}234

\item\rule{0.5\linewidth}{1ex} \test
\xx{\rule{0.5\linewidth}{1ex}}234

\begin{itemize}
\item\rule{0.15\linewidth}{1ex} \test
\xx{\rule{0.15\linewidth}{1ex}}234

\item\rule{0.3\linewidth}{1ex} \test
\xx{\rule{0.3\linewidth}{1ex}}234

\item\rule{0.5\linewidth}{1ex} \test
\xx{\rule{0.5\linewidth}{1ex}}234
\end{itemize}
\end{itemize}


\test
\fourtuxx
{\includegraphics[width=0.2\linewidth]{example-image}}
{\includegraphics[width=0.2\linewidth]{example-image}}
{\includegraphics[width=0.2\linewidth]{example-image}}
{\includegraphics[width=0.2\linewidth]{example-grid-100x100pt}}

\test
\fourtuxx
{\includegraphics[width=\dimexpr0.5\linewidth-1em\relax]{example-image}}
{\includegraphics[width=\dimexpr0.5\linewidth-1em\relax]{example-image}}
{\includegraphics[width=\dimexpr0.5\linewidth-1em\relax]{example-image}}
{\includegraphics[width=0.45\linewidth]{example-grid-100x100pt}}

The above image width is max, try more 0.01pt will get:
\fourtuxx
{\includegraphics[width=\dimexpr0.5\linewidth-1em+0.01pt\relax]{example-image-a}}
{}{}{}

tikz画带圈数字
\quan{100}
\foreach \i in {0,1,...,100} { \quan\i}


$\frac12+\underbrace{\frac12+\cdots+\frac12}_n$ 

$\dfrac12+\underbrace{\dfrac12+\cdots+\dfrac12}_n$ 
 
竖式除法longdiv
 $\longdiv{480632}{73}$
  $\longdiv{480632}{74}$


% \begin{split}
% &\underline {\ \ \ \ \ \ 1110}\
% 1011\big)&1100000\
% &\underline{1011\ \ \ \ \ \ }\
% &\ \ 1110\
% &\ \ \underline{1011\ \ \ \ }\
% &\ \ \ \ 1010\
% &\ \ \ \ \underline{1011\ \ }\
% &\ \ \ \ \ \ \ \ \ 010
% \end{split}
%
 长括号
 $\left(\frac{1}{\frac{2}{3}}\right)$
 $\left.\dfrac{dy}{dx}\right|_{x=0}$
 $\frac{df}{dx}\bigg|_{x = x_0} $
 
 在需要加粗的数学公式前使用\textbackslash bm命令,例如:
\bm{${{-0.628}^{**}}$}

\end{CJK}
\end{document}
